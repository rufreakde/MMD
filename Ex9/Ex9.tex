\documentclass[11pt,a4paper]{scrartcl}
\usepackage[top=3cm,bottom=3cm,left=2cm,right=2cm]{geometry} % Seitenränder einstellen
\usepackage[utf8]{inputenc} % Umlaute im Text
\usepackage[english]{babel} % Worttrennung nach der neuen Rechtschreibung und deutsche Bezeichnungen
\usepackage[dvipsnames]{xcolor} % Farbe in Dokument
\parindent 0pt % kein Einrücken bei neuem Absatz
\usepackage{amsmath} % zusätzliche mathematische Umgebungen
\usepackage{amssymb} % zusätzliche mathematische Symbole
%\usepackage{bbold} % zusätzliche mathematische Symbole
\usepackage{ upgreek }
\usepackage{units} % schöne Einheiten und Brüche
\usepackage{icomma} % kein Leerzeichen bei 1,23 in Mathe-Umgebung
\usepackage{wrapfig} % von Schrift umflossene Bilder und Tabellen
\usepackage{picinpar} % Objekt in Fließtext platzieren (ähnlich zu wrapfig)
\usepackage{scrhack} % verbessert andere Pakete, bessere Interaktion mit KOMA-Skript
\usepackage{float} % bessere Anpassung von Fließobjekten
\usepackage{pgf} % Makro zur Erstellung von Graphiken
\usepackage{tikz} % Benutzeroberfläche für pgf
\usepackage[margin=10pt,font=small,labelfont=bf,labelsep=endash,format=plain]{caption} % Einstellungen für Tabellen und Bildunterschriften

\usepackage{graphicx}
\graphicspath{ {U5_Ex5/} }

\usepackage{listings}
\usepackage{subcaption} % Unterschriften für mehrere Bilder
\usepackage{enumitem} % no indentation at description environment
\usepackage[onehalfspacing]{setspace} % Änderung des Zeilenabstandes (hier: 1,5-fach)
\usepackage{booktabs} % Einstellungen für schönere Tabellen
\usepackage{graphicx} % Einfügen von Grafiken -> wird in master-file geladen
\usepackage{url} % URL's (z.B. in Literatur) schöner formatieren
\usepackage[pdftex]{hyperref} % Verweise innerhalb und nach außerhalb des PDF; hyperref immer als letztes Paket einbinden

% define bordermatrix with brackets

\makeatletter
\def\bbordermatrix#1{\begingroup \m@th
  \@tempdima 4.75\p@
  \setbox\z@\vbox{%
    \def\cr{\crcr\noalign{\kern2\p@\global\let\cr\endline}}%
    \ialign{$##$\hfil\kern2\p@\kern\@tempdima&\thinspace\hfil$##$\hfil
      &&\quad\hfil$##$\hfil\crcr
      \omit\strut\hfil\crcr\noalign{\kern-\baselineskip}%
      #1\crcr\omit\strut\cr}}%
  \setbox\tw@\vbox{\unvcopy\z@\global\setbox\@ne\lastbox}%
  \setbox\tw@\hbox{\unhbox\@ne\unskip\global\setbox\@ne\lastbox}%
  \setbox\tw@\hbox{$\kern\wd\@ne\kern-\@tempdima\left[\kern-\wd\@ne
    \global\setbox\@ne\vbox{\box\@ne\kern2\p@}%
    \vcenter{\kern-\ht\@ne\unvbox\z@\kern-\baselineskip}\,\right]$}%
  \null\;\vbox{\kern\ht\@ne\box\tw@}\endgroup}
\makeatother

% make Titel
\title{Mining massive Datasets WS 2017/18}
\subtitle{Problem Set 8}
\author{Rudolf Chrispens, Marvin Klaus, Daniela Schacherer}

\begin{document}

\maketitle

\section*{Exercise 01}
MISSING

\section*{Exercise 02}
MISSING

\section*{Exercise 03}
	\begin{verbatim}
	{1, 2, 3} {2, 3, 4} {3, 4, 5} {4, 5, 6}
	{1, 3, 5} {2, 4, 6} {1, 3, 4} {2, 4, 5}
	{3, 5, 6} {1, 2, 4} {2, 3, 5} {3, 4, 6}
	
	Threshold: 4		PCY Algorithm we use a has table with 11 buckets
	\end{verbatim}
\begin{itemize}

\begin{table}[h]
\centering
\caption{3a) Items}
\label{my-label}
\begin{tabular}{lllllll}
Items & 1 & 2 & 3 & 4 & 5 & 6 \\
Support & 4 & 6 & 8 & 8 & 6 & 4
\end{tabular}
\end{table}

\begin{table}[h]
\centering
\caption{3a) Itempairs}
\label{my-label}
\begin{tabular}{lll}
Baskets & pairs           & support \\
1,2,3   & (1,2)(1,3)(2,3) & 2/3/3   \\
2,3,4   & (2,3)(2,4)(3,4) & 3/4/4   \\
3,4,5   & (3,4)(3,5)(4,5) & 4/4/3   \\
4,5,6   & (4,5)(4,6)(5,6) & 3/3/2   \\
1,3,5   & (1,3)(1,5)(3,5) & 3/1/4   \\
2,4,6   & (2,4)(2,6)(4,6) & 4/1/3   \\
1,3,4   & (1,3)(1,4)(3,4) & 3/2/4   \\
2,4,5   & (2,4)(2,5)(4,5) & 4/2/3   \\
3,5,6   & (3,5)(3,6)(5,6) & 4/2/2   \\
1,2,4   & (1,2)(1,4)(2,4) & 2/2/4   \\
2,3,5   & (2,3)(2,5)(3,5) & 3/2/4   \\
3,4,6   & (3,4)(3,6)(4,6) & 4/2/3  
\end{tabular}
\end{table}
	
\begin{table}[h]
\centering
\caption{3b) Hashbucket}
\label{my-label}
\begin{tabular}{llllllllllll}
Bucket & 1                                                     & 2                                                                             & 3                                                                     & 4                                                             & 5 & 6 & 7 & 8 & 9 & 10 & 11 \\
Count  & 2                                                     & 10                                                                            & 11                                                                    & 12                                                            & 0 & 0 & 0 & 0 & 0 & 0  & 0  \\
Pairs  & \begin{tabular}[c]{@{}l@{}}(1,5)\\ (2,6)\end{tabular} & \begin{tabular}[c]{@{}l@{}}(1,2)\\ (5,6)\\ (1,4)\\ (2,5)\\ (3,6)\end{tabular} & \begin{tabular}[c]{@{}l@{}}(1,3)\\ (2,3)\\ (4,5)\\ (4,6)\end{tabular} & \begin{tabular}[c]{@{}l@{}}(2,4)\\ (3,4)\\ (3,5)\end{tabular} &   &   &   &   &   &    &   
\end{tabular}
\end{table}
	
	\item[c)] 2,3 and 4 are Frequent, because their count exceeds our threshold of 4.
	
	\item[d)]  The count of each pair element in the Bucket.
	
		
\end{itemize} 

\section*{Exercise 04}
MISSING

\end{document}
