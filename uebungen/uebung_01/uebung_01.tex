\documentclass{article}
\usepackage[utf8]{inputenc}
\usepackage{fancyhdr} % Required for custom headers
\usepackage{lastpage} % Required to determine the last page for the footer
\usepackage{amsmath}
\usepackage{extramarks} % Required for headers and footers
\usepackage{graphicx} % Required to insert images
\usepackage{lipsum} % Used for inserting dummy 'Lorem ipsum' text into the template
\usepackage{caption}
\usepackage{subcaption}
\usepackage{listings}
\usepackage{color}

% Margins
% Margins
\topmargin=-0.45in
\evensidemargin=0in
\oddsidemargin=0in
\textwidth=6.5in
\textheight=9.0in
\headsep=0.25in 
\headheight=48pt

\linespread{1.1} % Line spacing

% Set up the header and footer
\pagestyle{fancy}
%\lhead{\hmwkClass} % Top left header
%\chead{\hmwkTitle} % Top center header
%\rhead{\hmwkAuthorName} % Top right header
\fancyhead[L]{}
\fancyhead[C]{\textbf{\hmwkTitle} \\ \ \\}
\fancyhead[R]{Rudolf Chrispens \\ Marvin Klaus 3486809\\ Daniela }
\lfoot{\lastxmark} % Bottom left footer
\cfoot{} % Bottom center footer
\rfoot{Page\ \thepage\ of\ \pageref{LastPage}} % Bottom right footer
\renewcommand\headrulewidth{0.4pt} % Size of the header rule
\renewcommand\footrulewidth{0.4pt} % Size of the footer rule

\setlength\parindent{0pt} % Removes all indentation from paragraphs

%----------------------------------------------------------------------------------------
%	DOCUMENT STRUCTURE COMMANDS
%	Skip this unless you know what you're doing
%----------------------------------------------------------------------------------------

% Header and footer for when a page split occurs within a problem environment
\newcommand{\enterProblemHeader}[1]{
\nobreak\extramarks{#1}{#1 continued on next page\ldots}\nobreak
\nobreak\extramarks{#1 (continued)}{#1 continued on next page\ldots}\nobreak
}

% Header and footer for when a page split occurs between problem environments
\newcommand{\exitProblemHeader}[1]{
\nobreak\extramarks{#1 (continued)}{#1 continued on next page\ldots}\nobreak
\nobreak\extramarks{#1}{}\nobreak
}

\setcounter{secnumdepth}{0} % Removes default section numbers
\newcounter{homeworkProblemCounter} % Creates a counter to keep track of the number of problems

\newcommand{\homeworkProblemName}{}
\newenvironment{homeworkProblem}[1][Problem \arabic{homeworkProblemCounter}]{ % Makes a new environment called homeworkProblem which takes 1 argument (custom name) but the default is "Problem #"
\stepcounter{homeworkProblemCounter} % Increase counter for number of problems
\renewcommand{\homeworkProblemName}{#1} % Assign \homeworkProblemName the name of the problem
\section{\homeworkProblemName} % Make a section in the document with the custom problem count
\enterProblemHeader{\homeworkProblemName} % Header and footer within the environment
}{
\exitProblemHeader{\homeworkProblemName} % Header and footer after the environment
}

\newcommand{\problemAnswer}[1]{ % Defines the problem answer command with the content as the only argument
\noindent\framebox[\columnwidth][c]{\begin{minipage}{0.98\columnwidth}#1\end{minipage}} % Makes the box around the problem answer and puts the content inside
}

\newcommand{\homeworkSectionName}{}
\newenvironment{homeworkSection}[1]{ % New environment for sections within homework problems, takes 1 argument - the name of the section
\renewcommand{\homeworkSectionName}{#1} % Assign \homeworkSectionName to the name of the section from the environment argument
\subsection{\homeworkSectionName} % Make a subsection with the custom name of the subsection
\enterProblemHeader{\homeworkProblemName\ [\homeworkSectionName]} % Header and footer within the environment
}{
\enterProblemHeader{\homeworkProblemName} % Header and footer after the environment
}
   
%----------------------------------------------------------------------------------------
%	NAME AND CLASS SECTION
%----------------------------------------------------------------------------------------

\newcommand{\hmwkTitle}{Übungsblatt 1} % Assignment title
\newcommand{\hmwkClass}{Mining Massivs Datasets} % Course/class
\newcommand{\hmwkClassTime}{10:30am} % Class/lecture time
\newcommand{\hmwkAuthorName}{Rudolf Chrispens \newline Marvin Klaus 3486809\newline Daniela}

%----------------------------------------------------------------------------------------

\begin{document}

\begin{homeworkProblem}

\begin{homeworkSection}{(a)}
$1-(1-p)^n$
\end{homeworkSection}

\begin{homeworkSection}{(b)}
$\frac{n}{1-(1-p)^n}$
\end{homeworkSection}

\begin{homeworkSection}{(c)}
If \textit{(b)} is added $n$ times, you get the solution of \textit{(a)}
\end{homeworkSection}

\end{homeworkProblem}


\begin{homeworkProblem}

\begin{homeworkSection}{(a)}
\textit{join()}: Applied on a dataset with tuples, \textit{join} pairs all the same keys with their value. Input should be a dataset with \textit{(K, V)} and \textit{(K, W)} tuples and the output is a dataset with \textit{(K, (V, W))} tuples. Example under \textit{(a)}.
\textit{sort()}: I couldn't find a \textit{sort()} Transformation. But I describe \textit{sortByKey()} instead. This Transformation gets an dataset of \textit{(K, V)} tuples and sort them by key. You can sort them descending or ascending.
\textit{groupBy()}: I couldn't find the \textit{groupBy()} Transformation and describe the \textit{groupByKey()} Transformation instead. This Transformation gets a dataset of tuples and groups values of the same key together. So the input tuples are \textit{(K, V)} and the output is \textit{(K, Iterable<V>)}.
\end{homeworkSection}

\begin{lstlisting}
// Example data
rdd1 =  sc.parallelize([("foo", 1), ("bar", 2), ("baz", 3)])
rdd2 =  sc.parallelize([("foo", 4), ("bar", 5), ("bar", 6)])

// Example join:
rdd1.join(rdd2)

// Example sortByKey:
rdd1.sortByKey()

// Example groupByKey:
rdd1.groupByKey()
\end{lstlisting}

\begin{homeworkSection}{(b)}

\end{homeworkSection}

\end{homeworkProblem}

\begin{homeworkProblem}
\end{homeworkProblem}

\begin{homeworkProblem}

\begin{homeworkSection}{(a)}
Broadcast provides a fast way to send data to each desired \textit{node} once. In this context the \textit{node} is one dataset of the data we want to map.\\ 
Broadcast is a better solution than to just \textit{join} the data. Once its send to a machine with Broadcast it stays cached on this machine and you can access the Broadcast values on the \textit{nodes}. But be careful, don't modify the Broadcast data.\\
A \textit{task} is the function we want to execute in the map step.
\end{homeworkSection}

\begin{homeworkSection}{(b)}
Accumulators are used to count something up. For example (in the video) we need this to count failures in the application.\\
If we want to build a custom accumulator we have to implement the type of the accumulator. For example a Vector class to build an accumulator of type Vector.\\
The accumulator can only be used in the Master (Exception on workers), the reduce() function gathers data from all tasks.
\end{homeworkSection}

\begin{homeworkSection}{(c)}
The \textit{join()}, \textit{mapValues()} and \textit{reduceByKey()} all results in partitioned RDD’s. With partitioning there is less traffic over networks what makes it much faster.\\
In the pageRank example the links have a partitioner so that links with the same hash are on the same node. To build a custom one you have to implement a class that extends from Partitioner. The class need the variable \textit{numPartitions} and the functions \textit{getPartition()} and \textit{equals()}.
\end{homeworkSection}

\end{homeworkProblem}

\end{document}
