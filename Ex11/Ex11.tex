\documentclass[11pt,a4paper]{scrartcl}
\usepackage[top=3cm,bottom=3cm,left=2cm,right=2cm]{geometry} % Seitenränder einstellen
\usepackage[utf8]{inputenc} % Umlaute im Text
\usepackage[english]{babel} % Worttrennung nach der neuen Rechtschreibung und deutsche Bezeichnungen
\usepackage[dvipsnames]{xcolor} % Farbe in Dokument
\parindent 0pt % kein Einrücken bei neuem Absatz
\usepackage{amsmath} % zusätzliche mathematische Umgebungen
\usepackage{amssymb} % zusätzliche mathematische Symbole
%\usepackage{bbold} % zusätzliche mathematische Symbole
\usepackage{ upgreek }
\usepackage{units} % schöne Einheiten und Brüche
\usepackage{icomma} % kein Leerzeichen bei 1,23 in Mathe-Umgebung
\usepackage{wrapfig} % von Schrift umflossene Bilder und Tabellen
\usepackage{picinpar} % Objekt in Fließtext platzieren (ähnlich zu wrapfig)
\usepackage{scrhack} % verbessert andere Pakete, bessere Interaktion mit KOMA-Skript
\usepackage{float} % bessere Anpassung von Fließobjekten
\usepackage{pgf} % Makro zur Erstellung von Graphiken
\usepackage{tikz} % Benutzeroberfläche für pgf
\usepackage[margin=10pt,font=small,labelfont=bf,labelsep=endash,format=plain]{caption} % Einstellungen für Tabellen und Bildunterschriften

\usepackage{graphicx}
\graphicspath{ {U5_Ex5/} }

\usepackage{listings}
\usepackage{subcaption} % Unterschriften für mehrere Bilder
\usepackage{enumitem} % no indentation at description environment
\usepackage[onehalfspacing]{setspace} % Änderung des Zeilenabstandes (hier: 1,5-fach)
\usepackage{booktabs} % Einstellungen für schönere Tabellen
\usepackage{graphicx} % Einfügen von Grafiken -> wird in master-file geladen
\usepackage{url} % URL's (z.B. in Literatur) schöner formatieren
\usepackage[pdftex]{hyperref} % Verweise innerhalb und nach außerhalb des PDF; hyperref immer als letztes Paket einbinden

% define bordermatrix with brackets

\makeatletter
\def\bbordermatrix#1{\begingroup \m@th
  \@tempdima 4.75\p@
  \setbox\z@\vbox{%
    \def\cr{\crcr\noalign{\kern2\p@\global\let\cr\endline}}%
    \ialign{$##$\hfil\kern2\p@\kern\@tempdima&\thinspace\hfil$##$\hfil
      &&\quad\hfil$##$\hfil\crcr
      \omit\strut\hfil\crcr\noalign{\kern-\baselineskip}%
      #1\crcr\omit\strut\cr}}%
  \setbox\tw@\vbox{\unvcopy\z@\global\setbox\@ne\lastbox}%
  \setbox\tw@\hbox{\unhbox\@ne\unskip\global\setbox\@ne\lastbox}%
  \setbox\tw@\hbox{$\kern\wd\@ne\kern-\@tempdima\left[\kern-\wd\@ne
    \global\setbox\@ne\vbox{\box\@ne\kern2\p@}%
    \vcenter{\kern-\ht\@ne\unvbox\z@\kern-\baselineskip}\,\right]$}%
  \null\;\vbox{\kern\ht\@ne\box\tw@}\endgroup}
\makeatother

% make Titel
\title{Mining massive Datasets WS 2017/18}
\subtitle{Problem Set 11}
\author{Rudolf Chrispens, Marvin Klaus, Daniela Schacherer}

\begin{document}

\maketitle

\section*{Exercise 01}
\begin{itemize}
	\item[a)] Let $w$ be the size of the sliding window. One can then compute the arithmetic mean (remember the sum over the stream elements $p$ and $w$) over the first $w$ sized window. As the window slides for one position one can subtract the first element (which is not in the window any more) from the sum, add the new element and again divide by $w$. This will result in 3 operations per new stream element.  
	\item[b)]
	\begin{itemize}
		\item[1.] If $p_{new} = p-o+n < p \Rightarrow n < o$
		\item[2.] $p_{new} = \frac{p*w -o^2 + n^2}{w}$
	\end{itemize}
	\item[c)] Stream 1: 
	\begin{align*}
		var_{t=0} = 20 \\
		var_{t=1} = 88
	\end{align*}
	Stream 2: 
	\begin{align*}
		var_{t=0} = 20 \\
		var_{t=1} = 130
	\end{align*}
	To compute the variance over a sliding window we have to keep track of $w$, the sum of the values, and the sum of the squares of the values like explained in a). Then one can compute the variance by $\sum_{i=1}^n (x_i - \vec{x})^2 = [\sum_{i=1}^n x_i^2] - 1/n [\sum_{i=1}^n x_i]^2$
	\item[d)] 
	
\end{itemize}
\section*{Exercise 02}

\section*{Exercise 03}
\begin{align*}
M =
 \begin{bmatrix}
1/3 & 1/2 & 0 \\
1/3 & 0 & 1/2  \\
1/3 & 1/2 & 1/2
\end{bmatrix} \\
r^{(0)} = \left( \begin{array}{c}1/3\\1/3\\1/3\end{array} \right), 
r^{(1)} = \left( \begin{array}{c}0.2777\\0.2777\\0.444\end{array} \right), 
r^{(2)} = \left( \begin{array}{c}0.231\\0.315\\0.454\end{array} \right), 
r^{(3)} = \left( \begin{array}{c}0.235\\0.304\\0.461\end{array} \right)\\
\end{align*}
At this point $\Vert r^{(3)} \Vert_1$ is $0.0216$ which is smaller than $\frac{1}{12} = \epsilon$.

\section*{Exercise 04}

\lstset{
  basicstyle=\ttfamily,
  columns=fullflexible,
  frame=single,
  breaklines=true,
  postbreak=\mbox{\textcolor{red}{$\hookrightarrow$}\space},
}	
MapReduce:
\begin{lstlisting}
map(inputTuple):
	word = inputTuple[1]
	numberLocals = word.count("a", "e", "i", "o", "u")	 # count and return the number of vowels in a word
	return(numberLocals, word)
	
reduce(tuple, list):
	insertTupleInSortedList(tuple, list)
	
main(input):
	list = []
	m = map(input)
	reduce(m, list)
	print(list[:10])
\end{lstlisting}

Spark:
\begin{lstlisting}
def countVowels(word):
	vowels = 0
	vowels = word.count("a") + word.count("e") + word.count("i") + word.count("o") + word.count("u")
	return (word, vowels)
	
topElements = rdd.map(countVowels)\
			.sortByKey(ascending=False)
			.top(10)
\end{lstlisting}

\section*{Exercise 05}
It may happen (but it is very unlikely) that in the phase when the k initialized clusters are recomputed with the points assigned to it, two cluster centers collapse. Thus fewer than k clusters can be the output.

\section*{Exercise 06}


\section*{Exercise 07}
\begin{itemize}
	\item[a)] PCA is a method for compressing a dataset with a lot of dimensions into new data that captures the essence o the original data. The resulting dataset is typically shown in under 4 dimensions to preserve comprehensibility. PCA looks into the dimensions and if the variations are not as "strong" as in the other dimensions the get "reduced". In the end PCA delivers Principle Components ordered with the "importance". (PC1 most important, PCn n important)
	\item[b)] 
	\item[c)]
\end{itemize}

\end{document}
