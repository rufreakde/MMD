\documentclass[11pt,a4paper]{scrartcl}
\usepackage[top=3cm,bottom=3cm,left=2cm,right=2cm]{geometry} % Seitenränder einstellen
\usepackage[utf8]{inputenc} % Umlaute im Text
\usepackage[english]{babel} % Worttrennung nach der neuen Rechtschreibung und deutsche Bezeichnungen
\usepackage[dvipsnames]{xcolor} % Farbe in Dokument
\parindent 0pt % kein Einrücken bei neuem Absatz
\usepackage{amsmath} % zusätzliche mathematische Umgebungen
\usepackage{amssymb} % zusätzliche mathematische Symbole
%\usepackage{bbold} % zusätzliche mathematische Symbole
\usepackage{ upgreek }
\usepackage{units} % schöne Einheiten und Brüche
\usepackage{icomma} % kein Leerzeichen bei 1,23 in Mathe-Umgebung
\usepackage{wrapfig} % von Schrift umflossene Bilder und Tabellen
\usepackage{picinpar} % Objekt in Fließtext platzieren (ähnlich zu wrapfig)
\usepackage{scrhack} % verbessert andere Pakete, bessere Interaktion mit KOMA-Skript
\usepackage{float} % bessere Anpassung von Fließobjekten
\usepackage{pgf} % Makro zur Erstellung von Graphiken
\usepackage{tikz} % Benutzeroberfläche für pgf
\usepackage[margin=10pt,font=small,labelfont=bf,labelsep=endash,format=plain]{caption} % Einstellungen für Tabellen und Bildunterschriften

\usepackage{graphicx}
\graphicspath{ {U5_Ex5/} }

\usepackage{listings}
\usepackage{subcaption} % Unterschriften für mehrere Bilder
\usepackage{enumitem} % no indentation at description environment
\usepackage[onehalfspacing]{setspace} % Änderung des Zeilenabstandes (hier: 1,5-fach)
\usepackage{booktabs} % Einstellungen für schönere Tabellen
\usepackage{graphicx} % Einfügen von Grafiken -> wird in master-file geladen
\usepackage{url} % URL's (z.B. in Literatur) schöner formatieren
\usepackage[pdftex]{hyperref} % Verweise innerhalb und nach außerhalb des PDF; hyperref immer als letztes Paket einbinden

% define bordermatrix with brackets

\makeatletter
\def\bbordermatrix#1{\begingroup \m@th
  \@tempdima 4.75\p@
  \setbox\z@\vbox{%
    \def\cr{\crcr\noalign{\kern2\p@\global\let\cr\endline}}%
    \ialign{$##$\hfil\kern2\p@\kern\@tempdima&\thinspace\hfil$##$\hfil
      &&\quad\hfil$##$\hfil\crcr
      \omit\strut\hfil\crcr\noalign{\kern-\baselineskip}%
      #1\crcr\omit\strut\cr}}%
  \setbox\tw@\vbox{\unvcopy\z@\global\setbox\@ne\lastbox}%
  \setbox\tw@\hbox{\unhbox\@ne\unskip\global\setbox\@ne\lastbox}%
  \setbox\tw@\hbox{$\kern\wd\@ne\kern-\@tempdima\left[\kern-\wd\@ne
    \global\setbox\@ne\vbox{\box\@ne\kern2\p@}%
    \vcenter{\kern-\ht\@ne\unvbox\z@\kern-\baselineskip}\,\right]$}%
  \null\;\vbox{\kern\ht\@ne\box\tw@}\endgroup}
\makeatother

% make Titel
\title{Mining massive Datasets WS 2017/18}
\subtitle{Problem Set 10}
\author{Rudolf Chrispens, Marvin Klaus, Daniela Schacherer}

\begin{document}

\maketitle

\section*{Exercise 01}
Let the input to the hash functions and thus the elements in $S$ be binary strings $s$. Then possible hash functions for a bloom filter could be
\begin{itemize}
	\item h1 takes every third position of $s$ starting from position 0, treats them as a number and computes modulo 11
	\item h2 takes every third position of $s$ starting from position 1, treats them as a number and computes modulo 11
	\item h1 takes every third position of $s$ starting from position 2, treats them as a number and computes modulo 11
\end{itemize}

These hash functions are independent from each other as they use different elements of $s$ for computing the hash value.

\section*{Exercise 02}
\begin{itemize}
	\item[a)] The probability that a random element ($m=1$) gets hashed to a given bit in the bit array with $n=5$ can be computed by the following formula:
	\begin{align*}
		1-(1-\frac{1}{n})^{n(m/n)} = 1 - e^{-m/n} = 1 - e^{-1/5} = 0.1813
	\end{align*}
	For $h_1(x)$ each bit is equally likely to be hit. For $h_2(x)$ this is not the case as $2x+3$ always results in an odd number which will then be taken modulo 5. \\
	Bit array state: $\vert$ 1 $\vert$ 0 $\vert$ 0 $\vert$ 0 $\vert$ 1 $\vert$
	\item[b)] With $k=2, n=5$ and $m=1$ unknown the probability for false positives is:
	\begin{align*}
		(1-e^{-km/n})^k = (1-e^{-2/5})^2 = 0.109 
	\end{align*}
\end{itemize}

\section*{Exercise 03}
With the same formula as in Exercise 2b we receive with $n=8$ billion, $m=1$ billion and $k=3$
\begin{align*}
	(1-e^{-km/n})^k = (1-e^{-3*1/8})^3 = 0.0306
	\intertext{with $k=4$ we get}
	(1-e^{-4*1/8})^4 = 0.024
\end{align*}

\end{document}
