\documentclass[11pt,a4paper]{scrartcl}
\usepackage[top=3cm,bottom=3cm,left=2cm,right=2cm]{geometry} % Seitenränder einstellen
\usepackage[utf8]{inputenc} % Umlaute im Text
\usepackage[english]{babel} % Worttrennung nach der neuen Rechtschreibung und deutsche Bezeichnungen
\usepackage[dvipsnames]{xcolor} % Farbe in Dokument
\parindent 0pt % kein Einrücken bei neuem Absatz
\usepackage{amsmath} % zusätzliche mathematische Umgebungen
\usepackage{amssymb} % zusätzliche mathematische Symbole
%\usepackage{bbold} % zusätzliche mathematische Symbole
\usepackage{units} % schöne Einheiten und Brüche
\usepackage{icomma} % kein Leerzeichen bei 1,23 in Mathe-Umgebung
\usepackage{wrapfig} % von Schrift umflossene Bilder und Tabellen
\usepackage{picinpar} % Objekt in Fließtext platzieren (ähnlich zu wrapfig)
\usepackage{scrhack} % verbessert andere Pakete, bessere Interaktion mit KOMA-Skript
\usepackage{float} % bessere Anpassung von Fließobjekten
\usepackage{pgf} % Makro zur Erstellung von Graphiken
\usepackage{tikz} % Benutzeroberfläche für pgf
\usepackage[margin=10pt,font=small,labelfont=bf,labelsep=endash,format=plain]{caption} % Einstellungen für Tabellen und Bildunterschriften
\usepackage{listings}
\usepackage{subcaption} % Unterschriften für mehrere Bilder
\usepackage{enumitem} % no indentation at description environment
\usepackage[onehalfspacing]{setspace} % Änderung des Zeilenabstandes (hier: 1,5-fach)
\usepackage{booktabs} % Einstellungen für schönere Tabellen
\usepackage{graphicx} % Einfügen von Grafiken -> wird in master-file geladen
\usepackage{url} % URL's (z.B. in Literatur) schöner formatieren
\usepackage[pdftex]{hyperref} % Verweise innerhalb und nach außerhalb des PDF; hyperref immer als letztes Paket einbinden

% define bordermatrix with brackets

\makeatletter
\def\bbordermatrix#1{\begingroup \m@th
  \@tempdima 4.75\p@
  \setbox\z@\vbox{%
    \def\cr{\crcr\noalign{\kern2\p@\global\let\cr\endline}}%
    \ialign{$##$\hfil\kern2\p@\kern\@tempdima&\thinspace\hfil$##$\hfil
      &&\quad\hfil$##$\hfil\crcr
      \omit\strut\hfil\crcr\noalign{\kern-\baselineskip}%
      #1\crcr\omit\strut\cr}}%
  \setbox\tw@\vbox{\unvcopy\z@\global\setbox\@ne\lastbox}%
  \setbox\tw@\hbox{\unhbox\@ne\unskip\global\setbox\@ne\lastbox}%
  \setbox\tw@\hbox{$\kern\wd\@ne\kern-\@tempdima\left[\kern-\wd\@ne
    \global\setbox\@ne\vbox{\box\@ne\kern2\p@}%
    \vcenter{\kern-\ht\@ne\unvbox\z@\kern-\baselineskip}\,\right]$}%
  \null\;\vbox{\kern\ht\@ne\box\tw@}\endgroup}
\makeatother

% make Titel
\title{Mining massive Datasets WS 2017/18}
\subtitle{Problem Set 4}
\author{Rudolf Chrispens, Marvin, Daniela Schacherer}

\begin{document}

\maketitle

\section*{Exercise 01}
We use the formula $cos \phi = \frac{a*b}{||a|| * ||b||}$ where $\phi$ is the angle between the vectors $a$ and $b$. The weighting vector $w$ which is multiplied to $a$ and $b$ before calculating the cosine angle is $\left (\begin{array}{c} 1 \\ \alpha \\ \beta \end{array} \right)$. 
	\begin{itemize}
		\item[a)] Here we have $\alpha = 1$ and $ \beta = 1$. We receive the following cosine angles, which indicate that all three vectors point in almost the same direction:
		\begin{itemize}
			\item $\phi_{AB} = 0.13 ^\circ$
			\item $\phi_{AC} = 0.17 ^\circ$
			\item $\phi_{BC} = 0.28 ^\circ$
		\end{itemize}
		\item[b)] Here we have $\alpha = 0.01$ and $ \beta = 0.5$. The weighted vectors are thus
	
$\bordermatrix{
  & A	& B   & C  \cr
PS & 3.06 & 2.68 & 2.92 \cr
DS & 5 & 3.2 & 6.4 \cr
MMS & 3 & 2 & 3 \cr
}
$ 

We receive the following cosine angles:
		\begin{itemize}
			\item $\phi_{AB} = 7.74 ^\circ$
			\item $\phi_{AC} = 7.45 ^\circ$
			\item $\phi_{BC} = 14.26 ^\circ$
		\end{itemize}

		\item[c)] If we want to select $\alpha$ and $\beta$ as the invers proportional of the average in the respective component we receive $\alpha = \frac{1}{\frac{500+320+640}{3}} = \frac{1}{487}$ and $\beta = \frac{1}{\frac{6+4+6}{3}} = \frac{1}{5.34}$. \\
$\bordermatrix{
  & A	& B   & C  \cr
PS & 3.06 & 2.68 & 2.92 \cr
DS & 1.03 & 0.66 & 1.31 \cr
MMS & 1.12 & 0.75 & 1.12 \cr
}
$ 

With possible rounding errors during the calculation we receive for the angles:
		\begin{itemize}
			\item $\phi_{AB} = 6.01 ^\circ$
			\item $\phi_{AC} = 5.25 ^\circ$
			\item $\phi_{BC} = 10.67 ^\circ$
		\end{itemize}


	\end{itemize}
	
\section*{Exercise 01}	

	Consider a web shop that sells furniture and uses a recommendation system. 
	When a new user creates an account and likes one product, he will be presented with similar products on his next visit.

\begin{itemize}
\item[]How can a competitor - in principle - try to steal the valuable data for recommendation from this website?
	\begin{enumerate}
		\item -
	\end{enumerate}

\item[]Does this work better when the web shop implemented a content- based or a collaborative filtering system?
	\begin{enumerate}
		\item -
	\end{enumerate}

\item[]What data would the competitor be able to infer?
	\begin{enumerate}
		\item -
	\end{enumerate}
\item[]Would this technique have an impact on the recommendation system, i.e., would this attack create a bias on the data?
	\begin{enumerate}
		\item -
	\end{enumerate}
\item[]Why is this attack probably not viable in any case?
	\begin{enumerate}
		\item -
	\end{enumerate}
\end{itemize}
\end{document}

