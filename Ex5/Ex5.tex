\documentclass[11pt,a4paper]{scrartcl}
\usepackage[top=3cm,bottom=3cm,left=2cm,right=2cm]{geometry} % Seitenränder einstellen
\usepackage[utf8]{inputenc} % Umlaute im Text
\usepackage[english]{babel} % Worttrennung nach der neuen Rechtschreibung und deutsche Bezeichnungen
\usepackage[dvipsnames]{xcolor} % Farbe in Dokument
\parindent 0pt % kein Einrücken bei neuem Absatz
\usepackage{amsmath} % zusätzliche mathematische Umgebungen
\usepackage{amssymb} % zusätzliche mathematische Symbole
%\usepackage{bbold} % zusätzliche mathematische Symbole
\usepackage{units} % schöne Einheiten und Brüche
\usepackage{icomma} % kein Leerzeichen bei 1,23 in Mathe-Umgebung
\usepackage{wrapfig} % von Schrift umflossene Bilder und Tabellen
\usepackage{picinpar} % Objekt in Fließtext platzieren (ähnlich zu wrapfig)
\usepackage{scrhack} % verbessert andere Pakete, bessere Interaktion mit KOMA-Skript
\usepackage{float} % bessere Anpassung von Fließobjekten
\usepackage{pgf} % Makro zur Erstellung von Graphiken
\usepackage{tikz} % Benutzeroberfläche für pgf
\usepackage[margin=10pt,font=small,labelfont=bf,labelsep=endash,format=plain]{caption} % Einstellungen für Tabellen und Bildunterschriften

\usepackage{listings}
\usepackage{subcaption} % Unterschriften für mehrere Bilder
\usepackage{enumitem} % no indentation at description environment
\usepackage[onehalfspacing]{setspace} % Änderung des Zeilenabstandes (hier: 1,5-fach)
\usepackage{booktabs} % Einstellungen für schönere Tabellen
\usepackage{graphicx} % Einfügen von Grafiken -> wird in master-file geladen
\usepackage{url} % URL's (z.B. in Literatur) schöner formatieren
\usepackage[pdftex]{hyperref} % Verweise innerhalb und nach außerhalb des PDF; hyperref immer als letztes Paket einbinden

% define bordermatrix with brackets

\makeatletter
\def\bbordermatrix#1{\begingroup \m@th
  \@tempdima 4.75\p@
  \setbox\z@\vbox{%
    \def\cr{\crcr\noalign{\kern2\p@\global\let\cr\endline}}%
    \ialign{$##$\hfil\kern2\p@\kern\@tempdima&\thinspace\hfil$##$\hfil
      &&\quad\hfil$##$\hfil\crcr
      \omit\strut\hfil\crcr\noalign{\kern-\baselineskip}%
      #1\crcr\omit\strut\cr}}%
  \setbox\tw@\vbox{\unvcopy\z@\global\setbox\@ne\lastbox}%
  \setbox\tw@\hbox{\unhbox\@ne\unskip\global\setbox\@ne\lastbox}%
  \setbox\tw@\hbox{$\kern\wd\@ne\kern-\@tempdima\left[\kern-\wd\@ne
    \global\setbox\@ne\vbox{\box\@ne\kern2\p@}%
    \vcenter{\kern-\ht\@ne\unvbox\z@\kern-\baselineskip}\,\right]$}%
  \null\;\vbox{\kern\ht\@ne\box\tw@}\endgroup}
\makeatother

% make Titel
\title{Mining massive Datasets WS 2017/18}
\subtitle{Problem Set 4}
\author{Rudolf Chrispens, Marvin, Daniela Schacherer}

\begin{document}

\maketitle

\section*{Exercise 04}
	\begin{itemize}
	\item [a)] Load the data into Spark (RDD), converting each row into Ratings3 data structure. Split your dataset through random sample, 50\% into training and the remaining 50\% into test data.
	\begin{verbatim}
		siehe U5_Ex4.py
	\end{verbatim}	
	\item [b)] Use the Spark Mlib implementation of the Alternating Least Squares (ALS)4to train the ratings. Train your model using 10 latent factors and 5 iterations. Save the model to disk after training and submit the serialized model as part of the solution.
	\begin{verbatim}
		look into folder "./serialized/"
	\end{verbatim}	
	\item [c)] Predict the ratings of the test data and estimate the prediction quality through Mean Squared Error (MSE). Submit the obtained MSE as part of the solution.
	\begin{verbatim}
		siehe U5_Ex4.py
		Mean Squared Error = 1.4089626800206299
	\end{verbatim}	
\end{itemize}

\section*{Exercise 05}
Study the code of Albert Au Yeung for matrix factorization by GD (see slide ?References? on lecture 06).

\begin{itemize}
	\item [a)] Apply it to the utility matrix used in lecture 06 (slide "Recall: Utility Matrix"). Do you obtain the same matrices Q and P as shown in the lecture (slide "Latent Factor Models")? Submit as solutions your matrices Q, P and the full matrix R.

\begin{verbatim}	
Utility Matrix:
[[1 0 3 0 0 5 0 0 5 0 4 0]
 [0 0 5 4 0 0 4 0 0 2 1 3]
 [2 4 0 1 2 0 3 0 4 3 5 0]
 [0 2 4 0 5 0 0 4 0 0 2 0]
 [0 0 4 3 4 2 0 0 0 0 2 5]
 [1 0 3 0 3 0 0 2 0 0 4 0]]
P_Items_x_Features:
[[ 1.9131215   1.29077939  0.69386576]
 [-0.36571992  2.18788999  1.11153101]
 [ 2.07399878  0.12752533  1.4811241 ]
 [ 0.24444465  1.66461796  1.58692477]
 [ 0.45579533  1.4474079   1.46812956]
 [ 1.73770877  1.14169258  0.76518777]]
Q.T_Users_x_Features:
[[ 0.36644633  1.45147022  0.10310276 -0.09543998  0.05403266  2.17152558
   0.49735527  0.04048629  1.23370225  0.63101074  1.724804    1.61077307]
 [-0.14797647  0.34894672  1.64542336  1.46178083  1.7640161   0.34791981
   1.29415504  0.93109973  1.51234022  0.45076068  0.18410261  0.96770206]
 [ 0.77489518  0.64644431  1.02861432  0.65986469  1.13098052  0.41409709
   1.21251738  1.40630636  0.87901893  1.10528179  0.84746994  1.64667991]]
New_Utility_Matrix: R
[[ 1.04772461  3.6757977   3.03484693  2.16210581  3.16507532  4.89080778
   3.463294    2.25508736  4.92224102  2.55594999  4.12542583  5.47326927]
 [ 0.40354706  0.95116835  4.70563529  3.96658002  5.09683227  0.42732185
   3.9973268   3.58549026  3.83470134  1.98399656  0.71399017  3.35846964]
 [ 1.88885442  4.01231127  1.94717362  0.96581319  2.01214291  5.16143917
   2.99244048  2.28562155  4.05349477  3.0032584   4.85592728  5.90308521]
 [ 1.07295191  1.9615256   4.39653774  3.45713246  4.74440188  1.76711231
   4.20002342  3.79152438  4.21397752  2.65859057  2.07295065  4.61775624]
 [ 1.09048873  2.11570559  3.93873162  3.04105889  4.23830461  2.10130128
   3.87999505  3.4307745   4.04180258  2.56274318  2.29682484  4.55238191]
 [ 1.06077367  3.41527369  2.84481331  2.00797783  2.97326957  4.48755853
   3.26958928  2.20947145  4.54305735  2.45689111  3.85586924  5.16389207]]
   
   No we don't get the same results as in the lecture.
   
\end{verbatim}	
	\item [b)] Modify the code so that the error e is stored (or printed after each iteration), and plot the error over iterations (for matrix in a)), as well as the differences of the error in subsequent steps.
	\item [c)] Bonus, 3 points: In this code the learning rate "alpha" is a constant (alpha = 0.0002). How could you change the code to speed up the convergence? Explain your idea and implement your solution.
\end{itemize}
	
\end{document}
