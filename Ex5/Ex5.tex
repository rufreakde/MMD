\documentclass[11pt,a4paper]{scrartcl}
\usepackage[top=3cm,bottom=3cm,left=2cm,right=2cm]{geometry} % Seitenränder einstellen
\usepackage[utf8]{inputenc} % Umlaute im Text
\usepackage[english]{babel} % Worttrennung nach der neuen Rechtschreibung und deutsche Bezeichnungen
\usepackage[dvipsnames]{xcolor} % Farbe in Dokument
\parindent 0pt % kein Einrücken bei neuem Absatz
\usepackage{amsmath} % zusätzliche mathematische Umgebungen
\usepackage{amssymb} % zusätzliche mathematische Symbole
%\usepackage{bbold} % zusätzliche mathematische Symbole
\usepackage{units} % schöne Einheiten und Brüche
\usepackage{icomma} % kein Leerzeichen bei 1,23 in Mathe-Umgebung
\usepackage{wrapfig} % von Schrift umflossene Bilder und Tabellen
\usepackage{picinpar} % Objekt in Fließtext platzieren (ähnlich zu wrapfig)
\usepackage{scrhack} % verbessert andere Pakete, bessere Interaktion mit KOMA-Skript
\usepackage{float} % bessere Anpassung von Fließobjekten
\usepackage{pgf} % Makro zur Erstellung von Graphiken
\usepackage{tikz} % Benutzeroberfläche für pgf
\usepackage[margin=10pt,font=small,labelfont=bf,labelsep=endash,format=plain]{caption} % Einstellungen für Tabellen und Bildunterschriften

\usepackage{listings}
\usepackage{subcaption} % Unterschriften für mehrere Bilder
\usepackage{enumitem} % no indentation at description environment
\usepackage[onehalfspacing]{setspace} % Änderung des Zeilenabstandes (hier: 1,5-fach)
\usepackage{booktabs} % Einstellungen für schönere Tabellen
\usepackage{graphicx} % Einfügen von Grafiken -> wird in master-file geladen
\usepackage{url} % URL's (z.B. in Literatur) schöner formatieren
\usepackage[pdftex]{hyperref} % Verweise innerhalb und nach außerhalb des PDF; hyperref immer als letztes Paket einbinden

% define bordermatrix with brackets

\makeatletter
\def\bbordermatrix#1{\begingroup \m@th
  \@tempdima 4.75\p@
  \setbox\z@\vbox{%
    \def\cr{\crcr\noalign{\kern2\p@\global\let\cr\endline}}%
    \ialign{$##$\hfil\kern2\p@\kern\@tempdima&\thinspace\hfil$##$\hfil
      &&\quad\hfil$##$\hfil\crcr
      \omit\strut\hfil\crcr\noalign{\kern-\baselineskip}%
      #1\crcr\omit\strut\cr}}%
  \setbox\tw@\vbox{\unvcopy\z@\global\setbox\@ne\lastbox}%
  \setbox\tw@\hbox{\unhbox\@ne\unskip\global\setbox\@ne\lastbox}%
  \setbox\tw@\hbox{$\kern\wd\@ne\kern-\@tempdima\left[\kern-\wd\@ne
    \global\setbox\@ne\vbox{\box\@ne\kern2\p@}%
    \vcenter{\kern-\ht\@ne\unvbox\z@\kern-\baselineskip}\,\right]$}%
  \null\;\vbox{\kern\ht\@ne\box\tw@}\endgroup}
\makeatother

% make Titel
\title{Mining massive Datasets WS 2017/18}
\subtitle{Problem Set 4}
\author{Rudolf Chrispens, Marvin, Daniela Schacherer}

\begin{document}

\maketitle

\section*{Exercise 04}
	\begin{itemize}
	\item [a)] Load the data into Spark (RDD), converting each row into Ratings3 data structure. Split your dataset through random sample, 50\% into training and the remaining 50\% into test data.
	\begin{verbatim}
		siehe U5_Ex4.py
	\end{verbatim}	
	\item [b)] Use the Spark Mlib implementation of the Alternating Least Squares (ALS)4to train the ratings. Train your model using 10 latent factors and 5 iterations. Save the model to disk after training and submit the serialized model as part of the solution.
	\begin{verbatim}
		look into folder "./serialized/"
	\end{verbatim}	
	\item [c)] Predict the ratings of the test data and estimate the prediction quality through Mean Squared Error (MSE). Submit the obtained MSE as part of the solution.
	\begin{verbatim}
		siehe U5_Ex4.py
		Mean Squared Error = 1.4089626800206299
	\end{verbatim}	
\end{itemize}

\section*{Exercise 05}
Study the code of Albert Au Yeung for matrix factorization by GD (see slide ?References? on lecture 06).

\begin{itemize}
	\item [a)] Apply it to the utility matrix used in lecture 06 (slide "Recall: Utility Matrix"). Do you obtain the same matrices Q and P as shown in the lecture (slide "Latent Factor Models")? Submit as solutions your matrices Q, P and the full matrix R.

\begin{verbatim}	
Utility Matrix:
[[1 0 3 0 0 5 0 0 5 0 4 0]
 [0 0 5 4 0 0 4 0 0 2 1 3]
 [2 4 0 1 2 0 3 0 4 3 5 0]
 [0 2 4 0 5 0 0 4 0 0 2 0]
 [0 0 4 3 4 2 0 0 0 0 2 5]
 [1 0 3 0 3 0 0 2 0 0 4 0]]

P_Items_x_Features:
[[ 1.16482079  2.10927657  0.33377973]
 [-0.36529105  1.82670981  1.7490211 ]
 [ 2.23435165  0.99895337  0.16300664]
 [ 0.26892799  1.78827864  1.44666862]
 [ 0.98234642  0.65571352  1.81046416]
 [ 1.47180934  1.33842574  0.58534001]]
Q_Users_x_Features:
[[ 0.85874813  1.50577869  0.27737099 -0.06739603  0.20751181  0.61557246
   0.63068878 -0.12666912  0.97616569  0.93279814  1.78547743  1.98452877]
 [-0.15083256  0.59195287  1.04037346  0.96010801  1.2641353   1.97523311
   1.4244988   1.34004677  1.7287707   0.8344548   0.97725507  0.95529413]
 [ 0.4779314   0.34549956  1.67081246  1.29957806  1.71812591  0.07752532
   0.93788975  1.00716243  0.55460263  0.46795842 -0.1407221   1.22489854]]
New_Utility_Matrix:
[[ 0.84166392  3.1178754   3.07520619  2.38040184  3.48160063  4.9092209
   4.05234994  3.01515283  4.96862874  3.00283366  4.09409229  4.7354462 ]
 [ 0.24669178  1.13556468  4.72142554  4.05144734  5.23844462  3.51890801
   4.01214994  4.25569602  3.77138951  2.0020331   0.88681659  3.16248796]
 [ 1.84597661  4.01209113  1.93138243  1.02035656  2.00653249  3.36120825
   2.98507066  1.21979504  3.99846265  2.99406083  4.94267806  5.58809602]
 [ 0.65261913  1.9633461   4.35218243  3.57887476  4.80199072  3.80996529
   4.07382633  3.81934242  4.15636822  2.42007398  2.024191    4.01404971]
 [ 1.60996287  2.49286238  3.97960742  2.91618905  4.14336446  2.04024927
   3.25163377  2.5776853   3.09659938  2.31071615  2.13998442  4.79352892]
 [ 1.34178771  3.21073882  2.77869321  1.9465342   3.00305687  3.5950868
   3.3838239   2.19665276  4.07519209  2.76367158  3.85349543  4.91642036]]
   
   No we don't get the same results as in the lecture.
   
\end{verbatim}	
	\item [b)] Modify the code so that the error e is stored (or printed after each iteration), and plot the error over iterations (for matrix in a)), as well as the differences of the error in subsequent steps.
	\item [c)] Bonus, 3 points: In this code the learning rate "alpha" is a constant (alpha = 0.0002). How could you change the code to speed up the convergence? Explain your idea and implement your solution.
\end{itemize}
	
\end{document}
