\documentclass[11pt,a4paper]{scrartcl}
\usepackage[top=3cm,bottom=3cm,left=2cm,right=2cm]{geometry} % Seitenränder einstellen
\usepackage[utf8]{inputenc} % Umlaute im Text
\usepackage[english]{babel} % Worttrennung nach der neuen Rechtschreibung und deutsche Bezeichnungen
\usepackage[dvipsnames]{xcolor} % Farbe in Dokument
\parindent 0pt % kein Einrücken bei neuem Absatz
\usepackage{amsmath} % zusätzliche mathematische Umgebungen
\usepackage{amssymb} % zusätzliche mathematische Symbole
%\usepackage{bbold} % zusätzliche mathematische Symbole
\usepackage{units} % schöne Einheiten und Brüche
\usepackage{icomma} % kein Leerzeichen bei 1,23 in Mathe-Umgebung
\usepackage{wrapfig} % von Schrift umflossene Bilder und Tabellen
\usepackage{picinpar} % Objekt in Fließtext platzieren (ähnlich zu wrapfig)
\usepackage{scrhack} % verbessert andere Pakete, bessere Interaktion mit KOMA-Skript
\usepackage{float} % bessere Anpassung von Fließobjekten
\usepackage{pgf} % Makro zur Erstellung von Graphiken
\usepackage{tikz} % Benutzeroberfläche für pgf
\usepackage[margin=10pt,font=small,labelfont=bf,labelsep=endash,format=plain]{caption} % Einstellungen für Tabellen und Bildunterschriften
\usepackage{subcaption} % Unterschriften für mehrere Bilder
\usepackage{enumitem} % no indentation at description environment
\usepackage[onehalfspacing]{setspace} % Änderung des Zeilenabstandes (hier: 1,5-fach)
\usepackage{booktabs} % Einstellungen für schönere Tabellen
\usepackage{graphicx} % Einfügen von Grafiken -> wird in master-file geladen
\usepackage{url} % URL's (z.B. in Literatur) schöner formatieren
\usepackage[pdftex]{hyperref} % Verweise innerhalb und nach außerhalb des PDF; hyperref immer als letztes Paket einbinden

% define bordermatrix with brackets

\makeatletter
\def\bbordermatrix#1{\begingroup \m@th
  \@tempdima 4.75\p@
  \setbox\z@\vbox{%
    \def\cr{\crcr\noalign{\kern2\p@\global\let\cr\endline}}%
    \ialign{$##$\hfil\kern2\p@\kern\@tempdima&\thinspace\hfil$##$\hfil
      &&\quad\hfil$##$\hfil\crcr
      \omit\strut\hfil\crcr\noalign{\kern-\baselineskip}%
      #1\crcr\omit\strut\cr}}%
  \setbox\tw@\vbox{\unvcopy\z@\global\setbox\@ne\lastbox}%
  \setbox\tw@\hbox{\unhbox\@ne\unskip\global\setbox\@ne\lastbox}%
  \setbox\tw@\hbox{$\kern\wd\@ne\kern-\@tempdima\left[\kern-\wd\@ne
    \global\setbox\@ne\vbox{\box\@ne\kern2\p@}%
    \vcenter{\kern-\ht\@ne\unvbox\z@\kern-\baselineskip}\,\right]$}%
  \null\;\vbox{\kern\ht\@ne\box\tw@}\endgroup}
\makeatother

% make Titel
\title{Mining massive Datasets WS 2017/18}
\subtitle{Problem Set 2}
\author{Rudolf Chrispens, Marvin, Daniela Schacherer}

\begin{document}

\maketitle

	\section*{Exercise 01}
	See hand written solution. 
	\section*{Exercise 02}
	In order to also maintain information about which points are in which cluster one could store each Point p as \\
(\textit{vector sum of points in cluster, number of points in cluster, list of data points in the cluster}). \\

Pseudocode for step 3 and 4 of subroutine \textit{merge} from lecture 3:
	\begin{itemize}
		\item Find best pair, merge those two points/clusters and compute new cluster center\\
		d, (p,q), (ip, iq) = bestPair\\
		\textbf{sum} = p[0] + q[0] \\
		\textbf{count} = p[1] + q[1] \\
		\textbf{points} = p[2].append(q[2]) \\
		newCenter = (\textbf{sum}, \textbf{count}, \textbf{points}) \\

		\item Filter p and q from the inCluster
		\item Re-number centroid index in inCluster
		\item Add new cluster to the outCluster
		
	\end{itemize}

\section*{Exercise 04 - Pseudocode}
\paragraph{Step 1:}
Take a small sample of the data and cluster it in main memory.\\
In principle, any clustering method could be used, but as CURE is designed to handle oddly shaped clusters, it is often advisable to use a hierarchical method in which clusters are merged when they have a close pair of points.
\begin{tabbing}
Links \= Mitte \= Rechts \kill
\>	\#Take a Sample of the whole Dataset\\
\>	\#use hierarchical clustering algorithm to cluster the sampleData\\
\> \>	 sampleSize;\\
\> \> 	sampleData = textFile.takeSample(false, sampleSize);\\
\> \>	hierarchicalCluster = HclusterAlgorythm( sampleData );
\end{tabbing}
\paragraph{Step 2:}
Select a small set of points from each cluster to be representative points. \\
These points should be chosen to be as far from one another as possible, \\
using the method described in Section 7.3.2.
\begin{tabbing}
Links \= Mitte \= Rechts  \=Rechts2 \kill
\#Select small set of points of each cluster. \\
\#They should be as far as possible from each other\\
FOR EACH hCluster in hierarchicalCluster DO\\
\>\#Pick the first point at random\\
\>\>	repPoints.add( hCluster.takeSample(false, 1) );\\
\>WHILE repPoints.count() $<$ hierarchicalCluster.count() DO\\
\>\#Add the point whose minimum distance from the selected points is as largest; \\
\>furthestPoint;\\
\>FOR EACH point in hCluster DO\\
\> \>		If( furthestPoint.squared\_distance( repPoints ) $<$ 
			Point.squared\_distance( repPoints ) ) DO\\
\> \>	\>		furthestPoint = point;\\
\>repPoints.add( furthestPoint);\\
END;
\end{tabbing}
\paragraph{Step 3:}
Move each of the representative points a fixed fraction of the distance between its location and the centroid of its cluster. Perhaps 20\% is a good fraction to choose. Note that this step requires a Euclidean space, since otherwise, there might not be any notion of a line between two points.

\begin{tabbing}
Links \= Mitte \= Rechts \kill
\> FOR EACH repPoint in repPoints DO\\
\> \> 	Center = clusterCenters(repPoints);\\
\> \> 	repPoint.moveTo(Center,0.2);
\end{tabbing}

\paragraph{Step 4:}
The next phase of CURE is to merge two clusters if they have a pair of representative points, one from each cluster, that are sufficiently close. The user may pick the distance that defines ?close.? This merging step can repeat, until there are no more sufficiently close clusters.
\begin{tabbing}
Links \= Mitte \= Rechts  \=Rechts2 \kill
\> 	Threshold;\\
\> 	FOR EACH cluster in repPoints DO\\
\> \> 		IF (closestClusterFound = closest\_cluster(repPoints, cluster, Threshold):\\
\> \> 	\>		mergeCluster(closestClusterFound, cluster):
\end{tabbing}
\paragraph{Step 5:}
The last step of CURE is point assignment. Each point p is brought from secondary storage and compared with the representative points. We assign p to the cluster of the representative point that is closest to p.
\begin{tabbing}
Links \= Mitte \= Rechts  \=Rechts2 \kill
\> FOR EACH point in Dataset DO\\
\> \>	nearestCluster(point, repPoints).add(point);
\end{tabbing}

\end{document}

