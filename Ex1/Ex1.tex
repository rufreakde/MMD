\documentclass[11pt,a4paper]{scrartcl}
\usepackage[top=3cm,bottom=3cm,left=2cm,right=2cm]{geometry} % Seitenränder einstellen
\usepackage[T1]{fontenc}
\usepackage[utf8]{inputenc} % Umlaute im Text
\usepackage{inconsolata}
\usepackage{tgpagella}
\usepackage{tgadventor}
\usepackage{minted}
\usepackage{pythontex}
\usepackage{hyperref}
\usepackage[english]{babel} % Worttrennung nach der neuen Rechtschreibung und deutsche Bezeichnungen
\usepackage[dvipsnames]{xcolor} % Farbe in Dokument
\parindent 0pt % kein Einrücken bei neuem Absatz
\usepackage{amsmath} % zusätzliche mathematische Umgebungen
\usepackage{amssymb} % zusätzliche mathematische Symbole
\usepackage{bbold} % zusätzliche mathematische Symbole
\usepackage{units} % schöne Einheiten und Brüche
\usepackage{icomma} % kein Leerzeichen bei 1,23 in Mathe-Umgebung
\usepackage{wrapfig} % von Schrift umflossene Bilder und Tabellen
\usepackage{picinpar} % Objekt in Fließtext platzieren (ähnlich zu wrapfig)
\usepackage{scrhack} % verbessert andere Pakete, bessere Interaktion mit KOMA-Skript
\usepackage{float} % bessere Anpassung von Fließobjekten
\usepackage{pgf} % Makro zur Erstellung von Graphiken
\usepackage{tikz} % Benutzeroberfläche für pgf
\usepackage[margin=10pt,font=small,labelfont=bf,labelsep=endash,format=plain]{caption} % Einstellungen für Tabellen und Bildunterschriften
\usepackage{subcaption} % Unterschriften für mehrere Bilder
\usepackage{enumitem} % no indentation at description environment
\usepackage[onehalfspacing]{setspace} % Änderung des Zeilenabstandes (hier: 1,5-fach)
\usepackage{booktabs} % Einstellungen für schönere Tabellen
\usepackage{graphicx} % Einfügen von Grafiken -> wird in master-file geladen
\usepackage{url} % URL's (z.B. in Literatur) schöner formatieren
\usepackage[pdftex]{hyperref} % Verweise innerhalb und nach außerhalb des PDF; hyperref immer als letztes Paket einbinden

% define bordermatrix with brackets

\makeatletter
\def\bbordermatrix#1{\begingroup \m@th
  \@tempdima 4.75\p@
  \setbox\z@\vbox{%
    \def\cr{\crcr\noalign{\kern2\p@\global\let\cr\endline}}%
    \ialign{$##$\hfil\kern2\p@\kern\@tempdima&\thinspace\hfil$##$\hfil
      &&\quad\hfil$##$\hfil\crcr
      \omit\strut\hfil\crcr\noalign{\kern-\baselineskip}%
      #1\crcr\omit\strut\cr}}%
  \setbox\tw@\vbox{\unvcopy\z@\global\setbox\@ne\lastbox}%
  \setbox\tw@\hbox{\unhbox\@ne\unskip\global\setbox\@ne\lastbox}%
  \setbox\tw@\hbox{$\kern\wd\@ne\kern-\@tempdima\left[\kern-\wd\@ne
    \global\setbox\@ne\vbox{\box\@ne\kern2\p@}%
    \vcenter{\kern-\ht\@ne\unvbox\z@\kern-\baselineskip}\,\right]$}%
  \null\;\vbox{\kern\ht\@ne\box\tw@}\endgroup}
\makeatother

% make Titel
\title{Mining massive Datasets WS 2017/18}
\subtitle{Problem Set 1}
\author{Rudolph, Marvin, Daniela Schacherer}

\begin{document}
	\maketitle
	
	\begin{minted}{latex}
		\begin{pycode}
			lo, hi = 1, 6
			print(r"\begin{tabular}{c|c}")
				print(r"$m$ & $2^m$ \\ \hline")
				for m in range(lo, hi + 1):
				print(r"%d & %d \\" % (m, 2**m))
				print(r"\end{tabular}")
		\end{pycode}
	\end{minted}
	
	
	\begin{pycode}
		lo, hi = 1, 6
		print(r"\begin{tabular}{c|c}")
			print(r"$m$ & $2^m$ \\ \hline")
			for m in range(lo, hi + 1):
			print(r"%d & %d \\" % (m, 2**m))
			print(r"\end{tabular}")
	\end{pycode}
	

	\section*{Exercise 01}
	Given is a cluster of $n$ machines, each having a probability $p$ of failing. 
	\begin{itemize}
		\item[a)] The probability of one machine to not fail is $1-p$. \\
		The probability of ALL machines not failing is $n$ times $1-p$ which is $(1-p)^n$. \\
		The probability of at least one machine failing is the opposite event and thus $1 - (1-p)^n$.
		\item[b)] The probability $p_k$ of exactly $k$ machines failing can be described using the binomial distribution. The binomial distribution describes the discrete probabilities of the number of successes in a sequence of independent experiments. As we have independent machines in the cluster with the number of successes corresponding to a machine failing we can write: \\
		\begin{align*}
			p(k|p,n) = {n\choose k} p^k (1-p)^{n-k}
		\end{align*}
		$p^k$ is the probablity that $k$ machines fail which has to be multiplied to the probability that the other $n-k$ machines do not fail. The binomial coefficient is the combinatoric element and describes in which way $k$ elements can be chosen from $n$ elements. 
		\item[c)] 
		Zz.: $p_1 + p_2 + ... + p_n = 1 - (1-p)^n$ \\
		We have $p_1 = p_2 = ... = p_n = p = {n \choose k} p^k (1-p)^{n-k}$
		\begin{align*}
			p_1 + p_2 + ... + p_n &= \sum_{k=1}^n {n \choose k} p^k (1-p)^{n-k} \\
			\intertext{We can use the binomial theorem: $\sum_{k=0}^n {n\choose k} y^k x^{n-k} = (x+y)^n$ but have to subtract $p_0$ again}
			&=  \sum_{k=0}^n {n \choose k} p^k (1-p)^{n-k} - {n \choose 0} p^0 (1-p)^n \\
			&= ((1-p) + p)^n - (1-p)^n	\\		
			&= 1^n - (1-p)^ n \\
			&= 1-(1-p)^n			
		\end{algin*}
	\end{itemize}
	





\end{document}

