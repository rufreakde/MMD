\documentclass[11pt,a4paper]{scrartcl}
\usepackage[top=3cm,bottom=3cm,left=2cm,right=2cm]{geometry} % Seitenränder einstellen
\usepackage[utf8]{inputenc} % Umlaute im Text
\usepackage[english]{babel} % Worttrennung nach der neuen Rechtschreibung und deutsche Bezeichnungen
\usepackage[dvipsnames]{xcolor} % Farbe in Dokument
\parindent 0pt % kein Einrücken bei neuem Absatz
\usepackage{amsmath} % zusätzliche mathematische Umgebungen
\usepackage{amssymb} % zusätzliche mathematische Symbole
\usepackage{bbold} % zusätzliche mathematische Symbole
\usepackage{units} % schöne Einheiten und Brüche
\usepackage{icomma} % kein Leerzeichen bei 1,23 in Mathe-Umgebung
\usepackage{wrapfig} % von Schrift umflossene Bilder und Tabellen
\usepackage{picinpar} % Objekt in Fließtext platzieren (ähnlich zu wrapfig)
\usepackage{scrhack} % verbessert andere Pakete, bessere Interaktion mit KOMA-Skript
\usepackage{float} % bessere Anpassung von Fließobjekten
\usepackage{pgf} % Makro zur Erstellung von Graphiken
\usepackage{tikz} % Benutzeroberfläche für pgf
\usepackage[margin=10pt,font=small,labelfont=bf,labelsep=endash,format=plain]{caption} % Einstellungen für Tabellen und Bildunterschriften
\usepackage{subcaption} % Unterschriften für mehrere Bilder
\usepackage{enumitem} % no indentation at description environment
\usepackage[onehalfspacing]{setspace} % Änderung des Zeilenabstandes (hier: 1,5-fach)
\usepackage{booktabs} % Einstellungen für schönere Tabellen
\usepackage{graphicx} % Einfügen von Grafiken -> wird in master-file geladen
\usepackage{url} % URL's (z.B. in Literatur) schöner formatieren
\usepackage[pdftex]{hyperref} % Verweise innerhalb und nach außerhalb des PDF; hyperref immer als letztes Paket einbinden

% define bordermatrix with brackets

\makeatletter
\def\bbordermatrix#1{\begingroup \m@th
  \@tempdima 4.75\p@
  \setbox\z@\vbox{%
    \def\cr{\crcr\noalign{\kern2\p@\global\let\cr\endline}}%
    \ialign{$##$\hfil\kern2\p@\kern\@tempdima&\thinspace\hfil$##$\hfil
      &&\quad\hfil$##$\hfil\crcr
      \omit\strut\hfil\crcr\noalign{\kern-\baselineskip}%
      #1\crcr\omit\strut\cr}}%
  \setbox\tw@\vbox{\unvcopy\z@\global\setbox\@ne\lastbox}%
  \setbox\tw@\hbox{\unhbox\@ne\unskip\global\setbox\@ne\lastbox}%
  \setbox\tw@\hbox{$\kern\wd\@ne\kern-\@tempdima\left[\kern-\wd\@ne
    \global\setbox\@ne\vbox{\box\@ne\kern2\p@}%
    \vcenter{\kern-\ht\@ne\unvbox\z@\kern-\baselineskip}\,\right]$}%
  \null\;\vbox{\kern\ht\@ne\box\tw@}\endgroup}
\makeatother

% make Titel
\title{Mining massive Datasets WS 2017/18}
\subtitle{Problem Set 1}
\author{Rudolf Chrispens, Marvin, Daniela Schacherer}

\begin{document}

\maketitle

	\section*{Exercise 01}
	Given is a cluster of $n$ machines, each having a probability $p$ of failing. 
	\begin{itemize}
		\item[a)] The probability of one machine to not fail is $1-p$. \\
		The probability of ALL machines not failing is $n$ times $1-p$ which is $(1-p)^n$. \\
		The probability of at least one machine failing is the opposite event and thus $1 - (1-p)^n$.
		\item[b)] The probability $p_k$ of exactly $k$ machines failing can be described using the binomial distribution. The binomial distribution describes the discrete probabilities of the number of successes in a sequence of independent experiments. As we have independent machines in the cluster with the number of successes corresponding to a machine failing we can write: \\
		\begin{align*}
			p(k|p,n) = {n\choose k} p^k (1-p)^{n-k}
		\end{align*}
		$p^k$ is the probablity that $k$ machines fail which has to be multiplied to the probability that the other $n-k$ machines do not fail. The binomial coefficient is the combinatoric element and describes in which way $k$ elements can be chosen from $n$ elements. 
		\item[c)] 
		Zz.: $p_1 + p_2 + ... + p_n = 1 - (1-p)^n$ \\
		We have $p_1 = p_2 = ... = p_n = p = {n \choose k} p^k (1-p)^{n-k}$
		\begin{align*}
			p_1 + p_2 + ... + p_n &= \sum_{k=1}^n {n \choose k} p^k (1-p)^{n-k} \\
			\intertext{We can use the binomial theorem: $\sum_{k=0}^n {n\choose k} y^k x^{n-k} = (x+y)^n$ but have to subtract $p_0$ again}
			&=  \sum_{k=0}^n {n \choose k} p^k (1-p)^{n-k} - {n \choose 0} p^0 (1-p)^n \\
			&= ((1-p) + p)^n - (1-p)^n	\\		
			&= 1^n - (1-p)^ n \\
			&= 1-(1-p)^n			
		\end{align*}
	\end{itemize}
	
\section*{Exercise 02}
\begin{itemize}
	\item[a1)]
		-join() - TRANSFORMATION	
		Input:
		otherDataset, [numTasks]
		
		Output:
		Returns a dataset with "Key/(V1,V2)" pairs.
		
		Code Example:
    		rdd1 = sc.parallelize([("foo", 1), ("bar", 2), ("baz", 3)])
    		rdd2 = sc.parallelize([("foo", 4), ("bar", 5), ("bar", 6)])
    		rdd1.join(rdd2)
	\item[a2)]
    		-sort() - TRANSFORMATION - Could not find sort() in reference used sortByKey() instead
    		-https://spark.apache.org/docs/2.2.0/rdd-programming-guide.html#rdd-operations
    		Input:
		[ascending], [numTasks
		
		Output:
		When called on a dataset of (K, V) pairs where K implements Ordered, returns a dataset of (K, V) pairs sorted by keys in ascending or 			descending order, as specified in the boolean ascending argument.
		
		Code Example:
    		names = sc.textFile(sys.argv[1])
    		filtered_rows = names.filter(lambda line: "Count" not in line).map(lambda line: line.split(","))
    		filtered_rows.map(lambda n: (str(n[1]), int(n[4]))).sortByKey().collect()
	\item[a2)]
    		-groupby() - TRANSFORMATION - Could not find groupby() in reference used groupByKey() instead
    		-https://spark.apache.org/docs/2.2.0/rdd-programming-guide.html#rdd-operations
    		Input:
		[ascending], [numTasks]
		
		Output:
		Returns a dataset with "Key/Value" Pairs sorted ascending or descending.
		
		Code Example:
    		lines = spark.read.text(sys.argv[1]).rdd.map(lambda r: r[0])
    		words = lines.flatMap(lambda x: x.split(' '))
    		words.reduceByKey(lambda x, y: x + y, 5)
    		words.groupByKey(5)
    
	\item[b1.1)]
		-NOTE
		All the tested source code is in U1_Ex2.py
	
		-INTERSECTION
		Input:
		[RDD]
		
		Output:
		Returns a RDD with the intersecting elements of two datasets.
		
		Code example:
		intersectRDD1 = sc.parallelize(range(1, 10))
    		intersectRDD2 = sc.parallelize(range(5, 15))
    		intersect = intersectRDD1.intersection(intersectRDD2).collect()
    		print(intersect)
		
		exampleOutput:
		[8, 9, 5, 6, 7]
		
	\item[b1.2)]
		-DISTINCT
		Input:
		[numTasks]
		
		Output:
		Return a new dataset that contains the distinct elements of the source dataset.
		
		example Code:
		distinctRDD1 = sc.parallelize(range(1, 12))
    		distinctRDD2 = sc.parallelize(range(8, 20))
    		distinct = distinctRDD1.union(distinctRDD2).distinct().collect()
    		print(distinct)
		
		exampleOutput:
		[8, 16, 1, 9, 17, 2, 10, 18, 3, 11, 19, 4, 12, 5, 13, 6, 14, 7, 15]
		
	\item[b1.3)]
		-UNION
		Input:
		[RDD]
		
		Output:
		Return a new dataset that contains the union of the elements in the source dataset and the argument.
		
		example Code:
		unionRDD1 = sc.parallelize(range(1, 7))
    		unionRDD2 = sc.parallelize(range(3, 10))
    		union = unionRDD1.union(unionRDD2).collect()
    		print(union)
		
		exampleOutput:
		[1, 2, 3, 4, 5, 6, 3, 4, 5, 6, 7, 8, 9]
		
	\item[b2.1)]
		-COLLECT
		Input:
		NONE is called as a function on an RDD
		
		Output:
		Return all the elements of the dataset as an array at the driver program.
		
		example Code:
		collection = sc.parallelize([1, 2, 3, 4, 5]).flatMap(lambda x: [x, x, x]).collect()
    		print(collection)
		
		exampleOutput:
		[1, 1, 1, 2, 2, 2, 3, 3, 3, 4, 4, 4, 5, 5, 5]
		
	\item[b2.2)]
		-COUNT
		Input:
		NONE is called as a function on an RDD
		
		Output:
		Return all the number of elements of the dataset as an array at the driver program.
		
		example Code:
		names1RDD = sc.parallelize(["Daniela", "Marvin", "Rudolf", "Kevin", "Jaqueline"])
    		counts = names1RDD.count()
    		print(counts)
		
		exampleOutput:
		5
		
	\item[b2.3)]
		-FIRST
		Input:
		NONE is called as a function on an RDD
		
		Output:
		Return all the first element of the dataset as an array at the driver program.
		
		example Code:
		names2RDD = sc.parallelize(["Daniela", "Marvin", "Rudolf"])
   	 	first = names2RDD.first()
    		print(first)
		
		exampleOutput:
		Daniela
		
	\end{itemize}

\section*{Exercise 03}
	\begin{itemize}
		\item [a)] see comments in 01-03_kmeans.py
		\item[b)] see 01-03_kmeans.py
	\end{itemize}

\section*{Exercise 05}
	\begin{itemize}
		\item Version 1: [:] is missing in line 130 \\
    In line 130 the variable centroids and newCentroids would refer to the same instance. In the for-loop newCentroids is changed and a new instance with the same values is created with centroids = newCentroids[:] in line 157.
		\item Version 2: [:] is missing in line 157 \\
    In line 130 centroids and newCentroids will refer to different instances. In the for-loop newCentroids is changed and the variable centroids is in line 157 assigned to newCentroids, meaning they then refer to the same object. In every further for-loop newCentroids will be changed and then assigned to centroids although they are already the same instance.
		\item Version 3: [:] is missing in line 130 and line 157 \\
    In line 130 the variable centroids and newCentroids would refer to the same instance. In the for-loop newCentroids is changed which changes also centroids as they refer to the same object. The same is true for every further for-loop. One of the two variables is therefore needless.
	\end{itemize}

\end{document}

