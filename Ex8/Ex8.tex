\documentclass[11pt,a4paper]{scrartcl}
\usepackage[top=3cm,bottom=3cm,left=2cm,right=2cm]{geometry} % Seitenränder einstellen
\usepackage[utf8]{inputenc} % Umlaute im Text
\usepackage[english]{babel} % Worttrennung nach der neuen Rechtschreibung und deutsche Bezeichnungen
\usepackage[dvipsnames]{xcolor} % Farbe in Dokument
\parindent 0pt % kein Einrücken bei neuem Absatz
\usepackage{amsmath} % zusätzliche mathematische Umgebungen
\usepackage{amssymb} % zusätzliche mathematische Symbole
%\usepackage{bbold} % zusätzliche mathematische Symbole
\usepackage{units} % schöne Einheiten und Brüche
\usepackage{icomma} % kein Leerzeichen bei 1,23 in Mathe-Umgebung
\usepackage{wrapfig} % von Schrift umflossene Bilder und Tabellen
\usepackage{picinpar} % Objekt in Fließtext platzieren (ähnlich zu wrapfig)
\usepackage{scrhack} % verbessert andere Pakete, bessere Interaktion mit KOMA-Skript
\usepackage{float} % bessere Anpassung von Fließobjekten
\usepackage{pgf} % Makro zur Erstellung von Graphiken
\usepackage{tikz} % Benutzeroberfläche für pgf
\usepackage[margin=10pt,font=small,labelfont=bf,labelsep=endash,format=plain]{caption} % Einstellungen für Tabellen und Bildunterschriften

\usepackage{graphicx}
\graphicspath{ {U5_Ex5/} }

\usepackage{listings}
\usepackage{subcaption} % Unterschriften für mehrere Bilder
\usepackage{enumitem} % no indentation at description environment
\usepackage[onehalfspacing]{setspace} % Änderung des Zeilenabstandes (hier: 1,5-fach)
\usepackage{booktabs} % Einstellungen für schönere Tabellen
\usepackage{graphicx} % Einfügen von Grafiken -> wird in master-file geladen
\usepackage{url} % URL's (z.B. in Literatur) schöner formatieren
\usepackage[pdftex]{hyperref} % Verweise innerhalb und nach außerhalb des PDF; hyperref immer als letztes Paket einbinden

% define bordermatrix with brackets

\makeatletter
\def\bbordermatrix#1{\begingroup \m@th
  \@tempdima 4.75\p@
  \setbox\z@\vbox{%
    \def\cr{\crcr\noalign{\kern2\p@\global\let\cr\endline}}%
    \ialign{$##$\hfil\kern2\p@\kern\@tempdima&\thinspace\hfil$##$\hfil
      &&\quad\hfil$##$\hfil\crcr
      \omit\strut\hfil\crcr\noalign{\kern-\baselineskip}%
      #1\crcr\omit\strut\cr}}%
  \setbox\tw@\vbox{\unvcopy\z@\global\setbox\@ne\lastbox}%
  \setbox\tw@\hbox{\unhbox\@ne\unskip\global\setbox\@ne\lastbox}%
  \setbox\tw@\hbox{$\kern\wd\@ne\kern-\@tempdima\left[\kern-\wd\@ne
    \global\setbox\@ne\vbox{\box\@ne\kern2\p@}%
    \vcenter{\kern-\ht\@ne\unvbox\z@\kern-\baselineskip}\,\right]$}%
  \null\;\vbox{\kern\ht\@ne\box\tw@}\endgroup}
\makeatother

% make Titel
\title{Mining massive Datasets WS 2017/18}
\subtitle{Problem Set 8}
\author{Rudolf Chrispens, Marvin Klaus, Daniela Schacherer}

\begin{document}

\maketitle

\section*{Exercise 01}

\begin{itemize}
	\item[a)] The probability that C1 and C2 are matched?\\
	QuastionnairesCount = b ( $Q_1$, ... , $Q_b$ )\\
	QuestionCount = r \\
	QuestionCountSum = $b * r$\\
	MatchCondition = if($mQ_x$(r) == $fQ_x(r)$) //m = person1, f = person2, x = element of b\\
	$Probability-of-an-event-happening$ =  $\frac{Number\ of\ ways\ it\ can\ happen}{Total\ number\ of\ outcomes}$\\
	Propability that 1 Questionair matsches: $(\frac{1}{r})^r * b$\\
	
	HIER BRAINSTORME ICH EINFACH NICHT BEACHTEN :D
	
	\item[b)] The probability that exactly two (no matter which) questionnaires match, i.e. have
the same answers for both C1 and C2.
\end{itemize} 

\section*{Exercise 02}
MISSING

\section*{Exercise 03}
MISSING

\section*{Exercise 04}
\begin{itemize}
	\item BALANCE Algorithm:\\
		For each query, assign it to an advertiser with the
		largest unspent budget (i.e. largest BALANCE) A(x,y) B(x,z)\\
		Worst Case Szenarios:

	\item[a)] xyyy - AA\_\_ ( since there are 3 $y$ no optimal solution possible, $A$ has only 2 dollars.)
	
	\item[b)] xyyx - A\_AB (optimal solution is possible but not certain BAAB)
	
	\item[c)] yyxx - AABB (Optimal)
	
	\item[d)] xzyz - BBA\_ (optimal solution is possible but not certain ABAB)
	
	\item Since we used the worst case possible query assignment with the balance Algorithm we can show that only $c)$ 			would give an optimal solution. All the other queries have situations where an optimal solution is not certain.
		$c$'s optimal solution is because A and B cant steal from each other, because their budget is empty before the other 		can take anything from their common $x$.
		
\end{itemize} 

\section*{Exercise 05}
\begin{itemize}
	\item budget: 3 / ties in favor of lower index
	\item $A_1(Q_1,Q_2)$ $A_2(Q_2,Q_3)$ $A_3(Q_3,Q_4)$ $A_4(Q_1,Q_4)$
	\item Query: $Q_1, Q_2, Q_3, Q_3, Q_1, Q_2, Q_3, Q_1, Q_4, Q_1, Q_4$

	\item[a)] What is the sequence of advertisers that the BALANCE algorithm will yield? \\
			$A_1,A_2,A_3,A_2,A_4,A_1,A_3,A_4,A_3,A_1,A_4$\\
	What is the competitive ratio for this instance?\\
	$Competitive\ ratio = min_(all\ possible\ inputs)\frac{|M_(greedy)|}{(|M_(opt|))})$
	
	
	
	\item[b)] Rearrange the sequence of queries so that BALANCE results in a worse competitive ratio.\\
		Query: $Q_1, Q_1, Q_1, Q_1, Q_4, Q_4, Q_2, Q_2, Q_3, Q_3, Q_3$\\
		$A_1,A_4,A_1,A_4,A_3,A_3,A_2,A_2,A_2,A_3, \_\_$
\end{itemize} 	
	

\section*{Exercise 06}
MISSING

\section*{Exercise 07}
MISSING
\end{document}
